\documentclass[]{elsarticle} %review=doublespace preprint=single 5p=2 column
%%% Begin My package additions %%%%%%%%%%%%%%%%%%%
\usepackage[hyphens]{url}

  \journal{An awesome journal} % Sets Journal name


\usepackage{lineno} % add
\providecommand{\tightlist}{%
  \setlength{\itemsep}{0pt}\setlength{\parskip}{0pt}}

\bibliographystyle{elsarticle-harv}
\biboptions{sort&compress} % For natbib
\usepackage{graphicx}
\usepackage{booktabs} % book-quality tables
%%%%%%%%%%%%%%%% end my additions to header

\usepackage[T1]{fontenc}
\usepackage{lmodern}
\usepackage{amssymb,amsmath}
\usepackage{ifxetex,ifluatex}
\usepackage{fixltx2e} % provides \textsubscript
% use upquote if available, for straight quotes in verbatim environments
\IfFileExists{upquote.sty}{\usepackage{upquote}}{}
\ifnum 0\ifxetex 1\fi\ifluatex 1\fi=0 % if pdftex
  \usepackage[utf8]{inputenc}
\else % if luatex or xelatex
  \usepackage{fontspec}
  \ifxetex
    \usepackage{xltxtra,xunicode}
  \fi
  \defaultfontfeatures{Mapping=tex-text,Scale=MatchLowercase}
  \newcommand{\euro}{€}
\fi
% use microtype if available
\IfFileExists{microtype.sty}{\usepackage{microtype}}{}
\ifxetex
  \usepackage[setpagesize=false, % page size defined by xetex
              unicode=false, % unicode breaks when used with xetex
              xetex]{hyperref}
\else
  \usepackage[unicode=true]{hyperref}
\fi
\hypersetup{breaklinks=true,
            bookmarks=true,
            pdfauthor={},
            pdftitle={Short Paper},
            colorlinks=true,
            urlcolor=blue,
            linkcolor=magenta,
            pdfborder={0 0 0}}
\urlstyle{same}  % don't use monospace font for urls

\setcounter{secnumdepth}{0}
% Pandoc toggle for numbering sections (defaults to be off)
\setcounter{secnumdepth}{0}
% Pandoc header



\begin{document}
\begin{frontmatter}

  \title{Short Paper}
    \author[Some Institute of Technology]{Alice Anonymous\corref{c1}}
   \ead{alice@example.com} 
   \cortext[c1]{Corresponding Author}
    \author[Another University]{Bob Security}
   \ead{bob@example.com} 
  
      \address[Some Institute of Technology]{Department, Street, City, State, Zip}
    \address[Another University]{Department, Street, City, State, Zip}
  
  \begin{abstract}
  This is the abstract.
  
  It consists of two paragraphs.
  \end{abstract}
  
 \end{frontmatter}

\section{Introduction}\label{introduction}

The primary objective of this paper is to develop a low-cost system that
can record and categorize track flows through residential neighborhoods.

\section{Methods}\label{methods}

To motivate this we provide an example of using repeat photography to
classify images of a roadway for the presence of FedEx trucks and buses
using TensorFlow, and then demonstrate ways to analyze traffic based on
labeled images.

\subsection{AOI Selection}\label{aoi-selection}

Although not critical for sucess in this case, many classification tasks
can be improved by restricting the field of view to the area of interest
(AOI) that contains the most informative components of an image. In this
case any components of the image above the roadway provide no
substantive information for the classification task and might throw off
the classifier through uninformative changes in lighting, phenology
across seasons, or changes in camera placement.

To minimize these issues we apply a multi-stage process to first
identify the roadway, then mask out unnessesary image elements. In the
first stage, we identify and isolate the yellow road centerline using
color selection.

yellow line image

Then yellow lines are then coverted to greyscale, and smoothed. These
smoothed lines can then be used with a Canny Edge Detector {[}@
reference{]}.

grey scale smoothed image

Edges can be defined as the boundary between an object and its
background. In its most basic form edge detectors, like Sobel filters,
use kernals (moving windows) to calculate the difference between
adjacent pixels in both the X and Y axis. High gradient values can be
treated as lines, and low gradient values are dropped from
consideration. Canny edge detection {[}@ 1986 {]} goes a few steps
further to try to isolate the strongest and most continuous lines. In
canny, edges detected by the Sobel kernals are then thinned to be one
pixel wide, and then filtered by histeresis thresholding. Each line is
scored by its strength relative to neighboring lines. Then to avoid
noise or non-continuous edges, Histeresis thresholding is applied to
return only the most prominent and continuous lines. Thresholds are
chosen between the values of zero (no edge) and two fifty five (sharp
edge). Two thresholds are chosen manually, the first, where all edges
with values less than the minimum threshold are dropped completely from
consideration. The second upper threshold is more complex, edges with
values above the upper threshold are always included, but edges with
thresholds between the minimum and maximum thresholds are only included
if they touch a line that is above the maximum threshold. As result
canny edge detection flexibly identifies strong and continuous lines,
while removing ones that are potenitally the result of noise or are weak
and non-continuous. https://www.youtube.com/watch?v=sRFM5IEqR2w

canny example

\subsection{Tensor Classifier}\label{tensor-classifier}

\section{Results}\label{results}

\section{Discussion}\label{discussion}

\emph{Text based on elsarticle sample manuscript, see
\url{http://www.elsevier.com/author-schemas/latex-instructions\#elsarticle}}

\section{The Elsevier article class}\label{the-elsevier-article-class}

\paragraph{Installation}\label{installation}

If the document class \emph{elsarticle} is not available on your
computer, you can download and install the system package
\emph{texlive-publishers} (Linux) or install the LaTeX package
\emph{elsarticle} using the package manager of your TeX installation,
which is typically TeX Live or MikTeX.

\paragraph{Usage}\label{usage}

Once the package is properly installed, you can use the document class
\emph{elsarticle} to create a manuscript. Please make sure that your
manuscript follows the guidelines in the Guide for Authors of the
relevant journal. It is not necessary to typeset your manuscript in
exactly the same way as an article, unless you are submitting to a
camera-ready copy (CRC) journal.

\paragraph{Functionality}\label{functionality}

The Elsevier article class is based on the standard article class and
supports almost all of the functionality of that class. In addition, it
features commands and options to format the

\begin{itemize}
\item
  document style
\item
  baselineskip
\item
  front matter
\item
  keywords and MSC codes
\item
  theorems, definitions and proofs
\item
  lables of enumerations
\item
  citation style and labeling.
\end{itemize}

\section{Front matter}\label{front-matter}

The author names and affiliations could be formatted in two ways:

\begin{enumerate}
\def\labelenumi{(\arabic{enumi})}
\item
  Group the authors per affiliation.
\item
  Use footnotes to indicate the affiliations.
\end{enumerate}

See the front matter of this document for examples. You are recommended
to conform your choice to the journal you are submitting to.

\section{Bibliography styles}\label{bibliography-styles}

There are various bibliography styles available. You can select the
style of your choice in the preamble of this document. These styles are
Elsevier styles based on standard styles like Harvard and Vancouver.
Please use BibTeXÂ~to generate your bibliography and include DOIs
whenever available.

Here are two sample references: Feynman and Vernon Jr. (1963; Dirac
1953).

\section*{References}\label{references}
\addcontentsline{toc}{section}{References}

\hypertarget{refs}{}
\hypertarget{ref-Dirac1953888}{}
Dirac, P.A.M. 1953. ``The Lorentz Transformation and Absolute Time.''
\emph{Physica} 19 (1---12): 888--96.
doi:\href{https://doi.org/10.1016/S0031-8914(53)80099-6}{10.1016/S0031-8914(53)80099-6}.

\hypertarget{ref-Feynman1963118}{}
Feynman, R.P, and F.L Vernon Jr. 1963. ``The Theory of a General Quantum
System Interacting with a Linear Dissipative System.'' \emph{Annals of
Physics} 24: 118--73.
doi:\href{https://doi.org/10.1016/0003-4916(63)90068-X}{10.1016/0003-4916(63)90068-X}.

\end{document}


